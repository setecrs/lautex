%\usepackage{hyperref} %creates pdf outline

\usepackage{times} %font

\usepackage{indentfirst}

\usepackage{setspace}
\onehalfspacing

\setlength{\parindent}{2.5cm}

\hyphenpenalty=100000

\DeclareUnicodeCharacter{00AD}{}
\DeclareUnicodeCharacter{00A0}{}


\usepackage[explicit]{titlesec}
\titleformat{\section}
{\normalfont\bf}{\thesection}{1em}{\MakeUppercase{#1}}
\renewcommand{\thesection}{\Roman{section}} 
%\renewcommand{\thesubsection}{\thesection.\Roman{subsection}}
%\renewcommand{\thesubsubsection}{\thesubsection.\Roman{subsubsection}}

\usepackage{titletoc}
\dottedcontents{section}[1.5em]{\bfseries}{1.3em}{.6em}

\usepackage{fancyhdr}
\fancypagestyle{first}{
	\fancyhf{}
	\fancyhead[C]{
		\singlespacing
		\includegraphics{brasao.jpg}\\
		\textbf{SERVIÇO PÚBLICO FEDERAL\\
		MJ – POLÍCIA FEDERAL\\
		SUPERINTENDÊNCIA REGIONAL NO ESTADO DO RIO GRANDE DO SUL\\
		SETOR TÉCNICO-CIENTÍFICO}
	}
	\setlength{\headheight}{4.5cm}
	\renewcommand{\headrulewidth}{0pt}
}
\fancypagestyle{custom}{
	\fancyhf{}
	\fancyhead[C]{\MakeUppercase{\pfdocnamenum}}
	\fancyfoot[C]{--\thepage--}
	\setlength{\headheight}{15pt}
}
\pagestyle{custom}

\usepackage{nameref}
\newcommand*{\fullref}[1]{``\hyperref[{#1}]{\ref*{#1} \nameref*{#1}}'' (pág. \pageref{#1})}

\newenvironment{quesitos}
{\begin{quotation}\obeylines\itshape}
	{\end{quotation}}

\newenvironment{quesitosresp}
{\begin{quotation}\bf\itshape``}
	{''\end{quotation}}


\newenvironment{materialenv}
{\singlespacing\noindent\obeylines\raggedright}
{\onehalfspacing} 


\newcommand{\material}[1]{
	\noindent\fbox{
		\parbox{\textwidth}{
			#1
		}
	}
}

\usepackage[space]{grffile} %support space in filenames
\newcommand{\PlotFig}[1]{%
	\begingroup%
	\catcode`\_=12%support _ in filenames
	\includegraphics[width=\linewidth]{#1}%
	\endgroup}

\usepackage{caption}
\newcommand{\minifig}[3]{%
\fbox{
	\begin{minipage}{\linewidth/#3-3\fboxsep-2\fboxrule}% to keep image and caption on one page
		\PlotFig{#1}%
		\captionof{figure}{\detokenize{#2}}%
	\end{minipage}%
}%
}

\newcommand{\halffig}[2]{%
	\minifig{#1}{#2}{2}
}

\newcommand{\fullfig}[2]{%
	\minifig{#1}{#2}{1}
}
